% CREATED BY DAVID FRISK, 2018
\chapter{Introduction}
In this chapter, the main picture of the thesis is presented. It starts with the background section mentioning the context which laid the base for the work to be done. Then, the aim of the thesis is described which precisely presents the goals. Problem analysis and research questions further detail the aspects which are concentrated and the exact matter requiring resolution. The scope of the thesis is outlined in delimitation section describing the clear boundaries for the task. \\

\section{Background}
For any organization, there is a need to continuously analyze how their present business performance is correlating with the strategy. Key performance indicators (KPIs) are measures created at various hierarchical levels to record the performance and align it to the vision or goals of a company. The basis on what KPIs are created and what factors considered while creating them play a very prominent role in usefulness of KPIs. \\ 

“If you can’t measure it, you can’t improve it” says Peter Drucker (MacKenzie,2019) KPIs need to measure accurately how the day to day work is aligned to the company’s business strategy. The measurement used should be feasible and correctly depict the ground situation. The next part after measuring is the analysis. The values measured from KPIs should be analysed periodically to see the trends and find the problem area to improve on. New age data analytics tools and software are available to perform this task. \\  

The most important aspect is how KPIs are modelled into the performance measurement framework and to what extent an employee connects his/her targets to his team goals > departmental goals > business strategy. Successful technological firms like Google, precisely measure their progress of daily work and the targets/goals are spread accurately at all hierarchical levels using a particular framework which is improved continuously (Doerr,2019).\\

Therefore, it is evident that a lot of factors have to be considered to develop & sustain a successful KPI culture. Many firms are not reaching/exceeding their potential due to not adhering to the above aspects of creating, measuring, analysing and developing support structure for the successful usage of KPIs. KPIs are seen just as targets which top management enforces and employees strive to put up the required numbers. But as described above, there needs to be a better junction of many aspects for effective KPI usage.\\

Volvo Group is one of the world’s leading manufacturers of trucks, buses, construction equipment, marine and industrial engines. The Group also provides complete solutions for financing and service. The Volvo Group, with its headquarters in Gothenburg, employs about 100 000 people, has production facilities in 18 countries and sells its products in more than 190 markets. (1). Further, Volvo Group has 3 main divisions. Group Trucks Technology (GTT), Group Trucks Operations (GTO) and Group Trucks Purchasing (GTP)\\ 

The professionals in GTT work in research & development of vehicles, powertrain, components and service offering. One of the departments in GTT is Powertrain Engineering. Right now, the department is undergoing a reorganization in the last eight months which will continue further for almost a year. Therefore, both the top management and team at Powertrain department want to review to analyze KPIs driving their organization.\\

KPIs provide a measure of how day to day work of team members is serving the organizational strategy. Data regarding day to day developments are summarized, quantified and reported to top management who then decide upon further decisions based on priority and organizational strategy. Therefore, it is necessary that KPIs should be framed perfectly so as it is easy to measure and reflects reality for managers.\\ 

In order to measure business performance in the most effective way KPIs should be aligned with business goals and strategy. For a large organization with multiple units such as Volvo, each business unit of that organization is required to develop its own KPIs to meet its unique strategy, while it’s also important to define the top KPIs used in common for all the business units, for the executives to make evaluations from a holistic perspective. \\

%\section{Section} \label{Section_ref}
%\subsection{Subsection}
%\subsubsection{Subsubsection}
%\paragraph{Paragraph}
%\subparagraph{Subparagraph}

\section{Aim}
The aim of the thesis work is to analyse the effectiveness of present KPIs.  Also, discuss factors to be considered to build the support infrastructure for effective use of KPIs. Based on the comprehensive analysis of both the aspects mentioned, the solution for successful usage of KPIs is suggested.\\
\section{Problem Analysis and Research Questions}
Powertrain Engineering (PE) department at Volvo Group presently has 23 KPIs to measure the operations. The KPIs are further grouped based on the 4 values; Customer success, Trust and passion, Change and Performance.In the present KPI chart for 2018, numbers inform the monthly performance, trends indicate the cumulative results and red/green/yellow colors specify the scores achieved in relation to the targets. Many of these KPIs have been around for years and have been lacking the impact for effective driving of the business. The time has arrived to analyse their impact on present business scenario.\\

The present situation at PE-Sweden is very dynamic. The strategic priorities for Volvo GTT have been revised since last two years, the whole department is undergoing reorganization known as PE Next Gear where the organization structure, roles and responsibilities of employees has been changed. Agile transformation efforts are ongoing to bring in more flexibility. Due to all these reasons, the present KPIs designed in before need to be revised. Comprehensive analysis of the present KPIs is needed to understand the limitations and work on improvements in future. Also, the effect of all the organizational changes needs to be considered to provide a good structure and support infrastructure for KPIs.\\

Based on the problem scenario described, the following research questions are formulated.\\

\begin{enumerate}
    \item \textbf{What is the impact of present PE Sweden KPIs?}\\
    The intention is to understand what results and behaviour do the present KPI´s drive on the business.Thorough reflection of present KPI´s is the first step to drive the improvements further and point out the shortcoming areas.\\

    \item \textbf{What are the ideal characteristics of the KPIs, what improvements are needed in the existing KPIs?}\\
    The idea here is to study the ideal characteristics of KPIs as discussed in academics and reflect with the KPIs under study to identify the gap for improvements. \\

    \item \textbf{What supporting factors are needed to increase the effectiveness of KPI’s at Volvo?}\\
    Developing apt KPI´s with flawless characteristics is only one part of the issue. Effective support mechanisms are needed to make them more meaningful, effective and sustainable.\\
  \end{enumerate}
\section{Delimitations}
The duration of this thesis work is twenty weeks.Therefore, certain considerations need to be made to achieve the desired results. \\ 

The research only focuses on the KPIs used by Powertrain Engineering Sweden department. Although it contains the global powertrain engineering KPIs, the focus of improvement is only delimited to PE-Sweden site as the empirical data considered only from here. Study of KPIs is conducted more from a business impact perspective, hence restricting the detailed study of technical aspects. The work aims to deliver proposals for changes in KPIs and support mechanisms to improve the efficacy of KPIs, which means the actual implementation of these changes will not be done.\\
