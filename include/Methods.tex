% CREATED BY DAVID FRISK, 2018
\chapter{Methodology}

This section presents the research methodology followed to conduct this master thesis. It begins with the choice of research strategy, research design and the research methods. Finally, a brief description of how the whole thesis was conducted chronologically is mentioned.

\section{Research Strategy}
In the area of business research, two principal research strategies exist; qualitative and quantitative research. A qualitative research strategy emphasizes words rather than quantifications in the collection and analysis of data and implies an inductive approach where theory is an outcome of the performed research (Bryman and Bell,2011). In our thesis, we wish to examine the current KPIs and try to find improvements and plan to make them better. To achieve this a deep understanding of present KPI measurement systems, evaluation procedures and decision models are necessary to be studied. This level of comprehensive understanding and depth is best obtained by qualitative research methods. However, during some parts of the analysis few quantitative methods are used to calculate the impact of KPIs and also some quantitative analysis to translate the findings from survey.\\

\section{Research Design}
A choice of research design reflects decisions about the priority being given to a range of dimensions of the research process (Bryman and Bell,2011). For this thesis, the chosen research design is Case study approach which deals with complications and unique features present in the task. As a particular organization has a unique way of managing KPIs and the primary interest is in researching how KPIs are driving the business, case study approach is apt. Needless to say, a single case study is conducted here at only one company (Volvo Group) in one location (Lundby, Gothenburg)\\



\section{Research Methods}
The various research methods employed to perform this thesis are as follows:\\
\begin{enumerate}
    \item \textbf{Literature Study:}
A literature study is performed in order to acquire a greater knowledge of the topic. Mainly there are two kinds of literature which need to be studied. Firstly, it is essential to get more theoretical knowledge about KPI characteristics, measurement, evaluation and different support mechanisms needed to be used in conjunction with them. These works of literature can be found by searching various research articles in Chalmers Library & Google scholar. On the other hand, the literature present in the Volvo Group about present management such as how each KPIs are measured, information flow map, decision making structures, communication patterns also need to be considered. Along with this, the literature needed to design semi-structured interviews and effective survey are also studied. After the completion of interviews and survey, sufficient literature is also reviewed to analyse the findings and convert the data into useful knowledge. Overall, this particular research method is employed at various phases of the research\\
  
    \item \textbf{Semi-structured Interviews:}
    Semi-structured interviews have been employed to conduct the thesis work. This provides a perfect balance as compared to open ended interviews or pre-planned questionnaire. As Gill et. al (2008) opine, Semi-structured interviews consist of several key questions that help to define the areas to be explored, but also allows the interviewer or interviewee to diverge in order to pursue an idea or response in more detail. This satisfies the main intention of letting respondents speak openly and provide more information within the scope of research.The interviewees are the KPI owners, each of the 7 directors of PE-Sweden, the Vice-President of PE-Sweden and Operation excellence team members. The comprehensive data obtained from all the interviews presents the view of KPIs from its owners who are responsible for measuring and analysing data, directors who act upon the KPI figures and the Vice-President who is in one way the end customer of these KPIs.\\
  
    \item \textbf{Survey:}
    It is also important to consider the perspective of the employees who work on these KPIs on a daily basis. Therefore, an anonymous survey for all the employees at PE-Sweden is conducted. After developing a basic understanding of KPIs at Volvo and interviews of top management & KPI owners, this survey is created and sent. The survey is designed using “Google forms” and is propagated using email as a medium. Survey is comprised of Yes/No questions, Multiple choice, short answers, Kano questionnaire etc. to bring in more variety and extract credible insights. At the end, by evaluating the results of survey true nature of employee perspective on present KPI management can be gained.\\
\end{enumerate}
    
    
\section{Phases} 

The thesis work was conducted in different phases. There were multiple phases with recurrence of the various research methods mentioned above. This section explains the journey traversed. The phases are explained below:\\

\begin{itemize}
    \item \textbf{Phase 1}\\
        The work started with understanding the basic characteristics, how they make impact on organizations and what could go wrong while using them. After getting a general picture, the culture at Volvo GTT was studied.\\

    \item \textbf{Phase 2}\\
        This phase focused on taking up the KPIs at Volvo for 2018 and understanding the definitions of each. During this phase, empirical data was collected mainly from the internal website of Volvo. Data like strategic goals of the company, the definitions and owners of KPIs were gathered.\\

    \item \textbf{Phase3}\\
        To get more knowledge about how the KPIs were being measured and evaluated, first a set of semi structured interviews were conducted. Initially, a literature study on framing the interview was done. Then all the KPI owners and the owner of global KPIs were interviewed.\\

    \item \textbf{Phase 4}\\
        During this phase, it was necessary to get insights from the top management. From the knowledge from KPI owners, website and literature on how interviews should be structured, all the directors and the Vice President were interviewed.
A literature study was also conducted on how this qualitative data could be analysed to draw conclusions. Methods like Recursive Abstraction, Coding were used to analyse this data.
\\

    \item \textbf{Phase 5}\\
        After understanding the top management views, the next phase was to study the views of teams. There were around 800+ employees hence decision was made on conducting a survey. A literature study was done on conducting survey after which a survey was framed. After initial pilot testing the survey was sent out for all PE-Sweden employees.\\  

    \item \textbf{Phase 6}\\
        In the last phase of the project, the focus here was to analyze the empirical data and draw conclusions by comparing this to literature.\\ 

\end{itemize}
