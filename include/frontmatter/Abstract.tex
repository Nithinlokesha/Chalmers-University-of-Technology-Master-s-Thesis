
\thispagestyle{plain}			% Supress header 
\setlength{\parskip}{0pt plus 1.0pt}
\section*{Abstract}
Ways of measuring the performance in organizations have changed across the years adapting to the requirements of the industrial revolution. One tool that has proved to be efficient to measure business performance are the Key Performance Indicators (KPI). Academic authors mention that the management might know the ideal ways to frame their KPIs, but they fail to find out what is right for them due to practical constraints. Thus the effectiveness of these KPIs might fade.\\ 

To study this, a case study approach at Volvo Group is taken to enhance the effectiveness of present KPIs at powertrain engineering from a holistic perspective. Along with gathering related academic information, research methods like interviews, surveys etc, the empirical data was collected to get better understanding and tools like recursive abstraction, KANO and Analytical hierarchical process(AHP) were used to compare the data with literature to draw conclusions from the case.\\

Two major pillars that impact the effectiveness of KPIs are their characteristics and the support factors. This thesis work explicitly highlights the SMART model and Leading, lagging indicators for analyzing the characteristics of KPIs. For the support factors, the work focuses on Top management support, Strategic cascading, Visual communication and Performance measurement. From the analysis, impact score is calculated for each KPI, to find the ones that needed improvement, the characteristics that needed to be improved are explicitly mentioned and considering the support factors, a KPI board was suggested which is a combination of all four support factors. However, this board could be further tested and refined in the future to adapt to the culture within the organization.\\

